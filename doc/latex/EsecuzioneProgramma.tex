\chapter*{Esecuzione programma}
\label{cha_esecuzione}

\section*{Piattaforma}
\label{sec_piattaforma}
La piattaforma usata è un computer portatile Lenovo W510. Nella Tabella~\ref{tab_statpc} sono mostrati alcuni parametri rilevanti.

\begin{table}[htb]
  \caption{Plattaforma usata per i test\label{tab_statpc}}
  \centering
\begin{tabular}{ll}
  \toprule
  Sistema Operativo		& Ubuntu 11.10 Oneiric Ocelot \\
  Kernel			& 3.0.0-16-generic x86\_64 \\
  Processore			& Intel Core i7 Q820 @ 1.73GHz \\
  RAM				& 12GB \\
  \bottomrule
\end{tabular}
\end{table}

\section*{Compilazione ed esecuzione}
\label{sec_comp_exec}
Per compilare il software è presente un makefile, quindi è sufficiente digitare:
\begin{lstlisting}
$ make
\end{lstlisting}
Per eseguire il programma è sufficiente digitare:
\begin{lstlisting}
$ ./TSP -s seed -a algorithm -m map_file -t tour_file
\end{lstlisting}
dove \emph{seed} è il seme per inizializzare il random, nel caso venga specifica l'opzione \emph{-s} senza indicare un \emph{seed} esso sarà scelto autonomamente dall'applicazione, \emph{algorithm} rappresenta l'algoritmo da utilizzare fra quelli conosciuti dal software e infine \emph{map\_file} e \emph{tour\_file} rispettivamente il file contenente la mappa del problema e il file contenente il tour della soluzione trovata. Il programma produce anche un output minimale sul risultato ottenuto, di seguito è presente un esempio:
\begin{lstlisting}[breaklines=true]
$ ./TSP -m /home/mikol/TSP/map/ch130.tsp -t /home/mikol/TSP/sol/ch130.tour -a caso

ch130 37410
\end{lstlisting}

